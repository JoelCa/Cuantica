\documentclass[a4paper,11pt]{article}
\usepackage[left=1.5cm,top=2.0cm,right=1.5cm,bottom=2.0cm]{geometry}
\usepackage[T1]{fontenc}
\usepackage[utf8]{inputenc}
\usepackage{lmodern,bm}
\usepackage[spanish]{babel}
\usepackage{hyperref}
\usepackage{caption}
\usepackage{subcaption}
\usepackage{amssymb}
\usepackage{enumitem}
\usepackage{amsthm}
\usepackage{braket}
\usepackage{stackrel}
\usepackage{qcircuit}
\usepackage{mathtools}

\title{}
\author{}

\newtheorem*{theorem}{Teorema}
\newtheorem{theorem2}{Teorema2}

\newtheorem{lemma}[theorem2]{Lema}

\begin{document}

\maketitle 

\thispagestyle{empty}
\begin{center}
\section*{Introducción a  la computación cuántica y fundamentos de lenguajes de programación} 
\subsection*{Ejercicios resueltos de la práctica 2}
\subsubsection*{Joel Catacora}

\end{center}

\newpage{\pagestyle{empty}\cleardoublepage}

\newpage

\section*
{Ejercicios}

\subsection*{Ejercicio 10}

\begin{theorem}
Sea $\rho$ un operador de densidad. Luego, $tr(p^2) \leq 1$.

Además, $tr(\rho^2) = 1 \Leftrightarrow \rho$ está en un estado puro.
\end{theorem}

\begin{proof}

Probemos que $tr(p^2) \leq 1$.


\begin{align*}
  \rho^2 
  &= \big( \sum_{i} p_i \ket{\Psi_i}\bra{\Psi_i} \big) \big( \sum_{j} p_j \ket{\Psi_j}\bra{\Psi_j} \big) 
  \\ &=\sum_{i, j} p_i p_j\ket{\Psi_i}\braket{\Psi_i | \Psi_j}\bra{\Psi_j}
 \end{align*}


Entonces, 

\begin{align*}
  tr(\rho^2) 
  &= \sum_{i, j} p_i p_j|\braket{\Psi_i | \Psi_j}|^2
  \\ &\le \sum_{i, j} p_i p_j \braket{\Psi_i | \Psi_i} \braket{\Psi_j | \Psi_j}
       \tag*{Lema 1}
  \\ &= \sum_{i,j} p_ip_j 
  \\ &= \Big( \sum_{i}p_i \Big)^2 
  \\ &= 1 
\end{align*}

Luego, $tr(\rho^2) \le 1$.

Ahora, veamos que si el sistema está en un estado puro, entonces $tr(\rho^2) = 1$.

Tenemos que, $tr(\rho^2) = tr(\ket{\Psi}\braket{\Psi | \Psi}\bra{\Psi}) = tr(\ket{\Psi}\bra{\Psi}) = 1$.

Por ultimo, tenemos que probar que si $tr(\rho^2) = 1$, entonces $\rho$ está en un estado puro.

\begin{align*}
  & 1 = tr(\rho^2) = \sum_{i, j} p_i p_j|\braket{\Psi_i | \Psi_j}|^2 
  \le \sum_{i, j} p_i p_j \braket{\Psi_i | \Psi_i} \braket{\Psi_j | \Psi_j} = 1.
  \\ &\Rightarrow 
       \sum_{i, j} p_i p_j|\braket{\Psi_i | \Psi_j}|^2 = \sum_{i, j} p_i p_j \braket{\Psi_i | \Psi_i} \braket{\Psi_j | \Psi_j}
  \\ &\Rightarrow |\braket{\Psi_i | \Psi_j}|^2 = \braket{\Psi_i | \Psi_i} \braket{\Psi_j | \Psi_j} \tag*{Lema 1}
  \\ &\Rightarrow \forall i, j, \exists \alpha_{ij}, \ket{ \Psi_{i}} = \alpha_{ij} \ket{\Psi_{j}}
  \\ &\Rightarrow \forall i, \exists \beta_{i}, \ket{ \Psi_{i}} = \beta_{i} \ket{\Psi_{0}}, \; | \beta_{i}|^2 = 1
\end{align*}

Entonces, tenemos un conjunto de estados puros de la forma, $\{ p_{i}, \beta_{i} \ket{\Psi_{0}}\}$, donde $| \beta_{i}|^2 = 1$.

Veamos que, $\rho$ está en un estado puro.

\begin{align*}
  \rho 
  &= \sum_{i} p_i |\beta_i|^2 \ket{\Psi_0} \bra{\Psi_0}
  \\ &=  \sum_{i} p_i \ket{\Psi_0} \bra{\Psi_0}
  \\ &= \ket{\Psi_0} \bra{\Psi_0} \sum_{i} p_i
  \\ &= \ket{\Psi_0} \bra{\Psi_0}
\end{align*}

Por lo tanto, $\rho$ está en un estado puro.

\end{proof}



\begin{lemma}
(Desigualdad de Cauchy-Schwarz).
Sean $\ket{\Psi}, \ket{\Phi}$ qubits. Luego, $|\braket{\Psi | \Phi}|^2 \leq \braket{\Psi | \Psi} \braket{\Phi | \Phi}$.
Mientras que se da la igualdad si y solo si $\ket{\Psi}$ y $\ket{\Phi}$ son linealmente dependientes. 
\end{lemma}

\subsection*{Ejercicio 14}

\begin{enumerate}[label=\alph*)]

\item Calcularemos la matriz de densidad $\rho_1$ del conjunto de estados puros $\{ (\frac{1}{4}, \ket{0}), (\frac{3}{4}, \ket{+}) \}$.

\begin{align*}
  \rho_1 
  &= \frac{1}{4} \ket{0}\bra{0} + \frac{3}{4} \ket{+}\bra{+} 
  \\ &=  \frac{1}{4} \ket{0}\bra{0} + \frac{3}{8} \big( \ket{0}\bra{0} + \ket{0}\bra{1} + \ket{1}\bra{0} + \ket{1}\bra{1} \big)
  \\ &= \frac{5}{8} \ket{0}\bra{0} + \frac{3}{8}\ket{0}\bra{1} + \frac{3}{8}\ket{1}\bra{0} + \frac{3}{8}\ket{1}\bra{1}
  \\ &= \bigl(\begin{smallmatrix}
\frac{5}{8} & \frac{3}{8}\\ 
\frac{3}{8} & \frac{3}{8}
\end{smallmatrix}\bigr)
\end{align*}

Ahora, calcularemos la matriz de densidad $\rho_2$ del conjunto de estados puros $\{ (\frac{3}{4}, \ket{1}), (\frac{1}{4}, \ket{-}) \}$.

\begin{align*}
  \rho_2 
  &= \frac{3}{4} \ket{1}\bra{1} + \frac{1}{8} \big( \ket{0}\bra{0} + \ket{0}\bra{1} + \ket{1}\bra{0} + \ket{1}\bra{1} \big)
  \\ &= \frac{1}{8} \ket{0}\bra{0} + \frac{1}{8}\ket{0}\bra{1} + \frac{1}{8}\ket{1}\bra{0} + \frac{7}{8}\ket{1}\bra{1}
  \\ &= \bigl(\begin{smallmatrix}
\frac{1}{8} & \frac{1}{8}\\ 
\frac{1}{8} & \frac{7}{8}
\end{smallmatrix}\bigr)
\end{align*}

Calculemos la matriz de densidad $\rho$ del sistema compuesto por los sistemas anteriores.

\begin{align*}
  \rho 
  &= \rho_1 \otimes \rho_2  
  \\ &= \frac{5}{64} \ket{00}\bra{00} + \frac{5}{64} \ket{00}\bra{01} + \frac{5}{64}\ket{01}\bra{00} + \frac{35}{64} \ket{01}\bra{01}
  \\ & + \frac{3}{64} \ket{00}\bra{10} + \frac{3}{64} \ket{00}\bra{11} + \frac{3}{64} \ket{01}\bra{10} + \frac{21}{64} \ket{01}\bra{11}
  \\ & + \frac{3}{64} \ket{10}\bra{00} + \frac{3}{64} \ket{10}\bra{01} + \frac{3}{64} \ket{11}\bra{00} + \frac{21}{64} \ket{11}\bra{01}
  \\ & + \frac{3}{64} \ket{10}\bra{10} + \frac{3}{64} \ket{10}\bra{11} + \frac{3}{64} \ket{11}\bra{10} + \frac{21}{64} \ket{11}\bra{11}
  \\ &= 
\begin{pmatrix}
\frac{5}{64} & \frac{5}{64} & \frac{3}{64} & \frac{3}{64}\\ 
\frac{5}{64} & \frac{35}{64} & \frac{3}{64} & \frac{21}{64}\\ 
 \frac{3}{64}& \frac{3}{64} & \frac{3}{64} & \frac{3}{64}\\ 
\frac{3}{64} & \frac{21}{64} & \frac{3}{64} & \frac{21}{64}
\end{pmatrix}
\end{align*}

\item Consideremos la medición proyectiva $\{ P_0, P_1, P_2, P_3 \}$, donde $P_0 = \ket{00}\bra{00}$, $P_1=\ket{11}\bra{11}$, 
$P_2 = \ket{01}\bra{01}$, $P_3 = \ket{01}\bra{10}$.

\begin{itemize}
\item Caso 0.
  \begin{align*}
    p(0)
    &= tr( P_0^{\dag} P_0 \rho )
    \\ &=  tr( \ket{00}\bra{00} \rho )
    \\ &= tr \Big( \frac{5}{64}\ket{00}\bra{00} + \frac{5}{64} \ket{00}\bra{01} + \frac{3}{64} \ket{00}\bra{10} + 
         \frac{3}{64}\ket{00}\bra{11} \Big)
    \\ &= \frac{5}{64} tr (\ket{00}\bra{00}) + \frac{5}{64} tr(\ket{00}\bra{01}) + \frac{3}{64} tr(\ket{00}\bra{10}) +
         \frac{3}{64} tr(\ket{00}\bra{11})
    \\ &= \frac{5}{64}
  \end{align*}

  El estado del sistema luego de la medición es el siguiente.

  $\frac{P_0 \rho P_{0}^{\dag}}{p(0)} = \frac{\frac{5}{64} \ket{00}\bra{00}}{\frac{5}{64}} = \ket{00}\bra{00}$

\item Caso 1.

  \begin{align*}
    p(1)
    &= tr( P_1^{\dag} P_1 \rho )
    \\ &=  tr( \ket{11}\bra{11} \rho )
    \\ &= tr \Big( \frac{3}{64}\ket{11}\bra{00} + \frac{21}{64} \ket{11}\bra{01} + \frac{3}{64} \ket{11}\bra{10} + 
         \frac{21}{64}\ket{11}\bra{11} \Big)
    \\ &= \frac{21}{64}
  \end{align*}

  El estado del sistema luego de la medición es el siguiente.

  $\frac{P_1 \rho P_{1}^{\dag}}{p(1)} = \frac{\frac{21}{64} \ket{11}\bra{11}}{\frac{21}{64}} = \ket{11}\bra{11}$
  
\item Caso 2.

  \begin{align*}
    p(2)
    &= tr( P_2^{\dag} P_2 \rho )
    \\ &=  tr( \ket{01}\bra{01} \rho )
    \\ &= tr \Big( \frac{5}{64}\ket{01}\bra{00} + \frac{35}{64} \ket{01}\bra{01} + \frac{3}{64} \ket{01}\bra{10} + 
         \frac{21}{64}\ket{01}\bra{11} \Big)
    \\ &= \frac{35}{64}
  \end{align*}

  El estado del sistema luego de la medición es el siguiente.

  $\frac{P_2 \rho P_{2}^{\dag}}{p(2)} = \frac{\frac{35}{64} \ket{01}\bra{01}}{\frac{35}{64}} = \ket{01}\bra{01}$

\item Caso 3.

  \begin{align*}
    p(3)
    &= tr( P_3^{\dag} P_3 \rho )
    \\ &=  tr( \ket{10}\bra{10} \rho )
    \\ &= tr \Big( \frac{3}{64}\ket{10}\bra{00} + \frac{3}{64} \ket{10}\bra{01} + \frac{3}{64} \ket{10}\bra{10} + 
         \frac{3}{64}\ket{10}\bra{11} \Big)
    \\ &= \frac{3}{64}
  \end{align*}

  Luego, el estado del sistema luego de la medición es el siguiente.

  $\frac{P_3 \rho P_{3}^{\dag}}{p(3)} = \frac{\frac{3}{64} \ket{10}\bra{10}}{\frac{3}{64}} = \ket{10}\bra{10}$

\end{itemize}

\end{enumerate}

\subsection*{Descomposición de Schmidt}

Vamos a calcular la descomposición de Schmidt del siquiente qubit.

$\ket{\Psi} = \frac{\ket{00} + \ket{01} + \ket{11}}{\sqrt{3}}$

Para ello, reescribimos el estado utilizando una matriz.

Tenemos que, $\ket{\Psi} = \sum_{i,j} A_{i,j} \ket{ij}$, donde

$A = 
\begin{pmatrix}
1 & 1\\ 
0 & 1
\end{pmatrix}
$

Ahora, vamos a calcular la descomposición en valores singulares de $A$.

$
U =
\begin{pmatrix}
\frac{1+\frac{1}{2}\left ( -1 + \sqrt{5} \right )}{\sqrt{1 + \left ( 1 +\frac{1}{2} 
\left ( -1 + \sqrt{5} \right ) \right )^2}} 
& \frac{1+\frac{1}{2}\left ( -1 - \sqrt{5} \right )}{\sqrt{1 + \left ( 1 +\frac{1}{2} 
\left ( -1 - \sqrt{5} \right ) \right )^2}} 
\\ 
\frac{1}{\sqrt{1 + \left ( 1 +\frac{1}{2} 
\left ( -1 + \sqrt{5} \right ) \right )^2}} 
 
& \frac{1}{\sqrt{1 + \left ( 1 +\frac{1}{2} 
\left ( -1 - \sqrt{5} \right ) \right )^2}} 
\end{pmatrix}
$

$
D =
\begin{pmatrix}
\sqrt{\frac{1}{2}\left ( 3 + \sqrt{5} \right )}
& 0
\\ 
0
& \sqrt{\frac{1}{2}\left ( 3 - \sqrt{5} \right )}
\end{pmatrix}
$

$V = 
\begin{pmatrix}
\frac{-1 + \sqrt{5}}{2\sqrt{1 + \frac{1}{4}\left ( -1 + \sqrt{5}\right )^2}} 
& \frac{-1 - \sqrt{5}}{2\sqrt{1 + \frac{1}{4}\left ( -1 - \sqrt{5}\right )^2}} 
\\ 
\frac{1}{\sqrt{1 + \frac{1}{4}\left ( -1 + \sqrt{5}\right )^2}} 
& \frac{1}{\sqrt{1 + \frac{1}{4}\left ( -1 - \sqrt{5}\right )^2}} 
\end{pmatrix}$

Luego, $A = U D V^{\dag}$

Entonces, $\ket{\Psi} = \sum_{i,j,k} U_{j,i} D_{i,i} V^{\dag}_{i,k}\ket{ij}$.

Tomamos, $\ket{i_A} = \sum_{j} U_{j,i}\ket{j}$

\end{document}