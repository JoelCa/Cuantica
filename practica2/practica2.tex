\documentclass[a4paper,11pt]{article}
\usepackage[left=1.5cm,top=2.0cm,right=1.5cm,bottom=2.0cm]{geometry}
\usepackage[T1]{fontenc}
\usepackage[utf8]{inputenc}
\usepackage{lmodern,bm}
\usepackage[spanish]{babel}
\usepackage{hyperref}
\usepackage{caption}
\usepackage{subcaption}
\usepackage{amssymb}
\usepackage{enumitem}
\usepackage{amsthm}
\usepackage{braket}
\usepackage{stackrel}
\usepackage{qcircuit}
\usepackage{mathtools}

\title{}
\author{}

\newtheorem*{theorem}{Teorema}
\newtheorem{theorem2}{Teorema2}

\newtheorem{lemma}[theorem2]{Lema}

\begin{document}

\maketitle 

\thispagestyle{empty}
\begin{center}
\section*{Introducción a  la computación cuántica y fundamentos de lenguajes de programación} 
\subsection*{Ejercicios resueltos de la práctica 2}
\subsubsection*{Joel Catacora}

\end{center}

\newpage{\pagestyle{empty}\cleardoublepage}

\newpage

\section*
{Ejercicios}

\subsection*{Ejercicio 10}

\begin{theorem}
Sea $\rho$ un operador de densidad. Luego, $tr(p^2) \leq 1$.

Además, $tr(\rho^2) = 1 \Leftrightarrow \rho$ está en un estado puro.
\end{theorem}

\begin{proof}
Veamos el caso donde $\rho$ está en un estado puro.

Luego, $tr(\rho^2) = tr(\ket{\Psi}\braket{\Psi | \Psi}\bra{\Psi}) = tr(\ket{\Psi}\bra{\Psi}) = 1$.

Ahora, vemos el caso general.

\begin{align*}
  \rho^2 
  &= \big( \sum_{i} p_i \ket{\Psi_i}\bra{\Psi_i} \big) \big( \sum_{j} p_j \ket{\Psi_j}\bra{\Psi_j} \big) 
  \\ &=\sum_{i, j} p_i p_j\ket{\Psi_i}\braket{\Psi_i | \Psi_j}\bra{\Psi_j}
  \\ &= \sum_{i} p_i^2\ket{\Psi_i}\bra{\Psi_i} + 
       \sum_{i, j, i \neq j} p_i p_j\ket{\Psi_i}\braket{\Psi_i | \Psi_j}\bra{\Psi_j}
\end{align*}


Entonces, 

\begin{align*}
  tr(\rho^2) 
  &= \sum_{i} p_i^2 + 
    \sum_{i, j, i \neq j} p_i p_j|\braket{\Psi_i | \Psi_j}|^2
  \\ &\le \sum_{i} p_i^2 + 
       \sum_{i, j, i \neq j} p_i p_j               \tag*{Lema 1}
  \\ &= \sum_{i,j} p_ip_j 
  \\ &= \Big( \sum_{i}p_i \Big)^2 
  \\ &= 1 
\end{align*}

\end{proof}

\begin{lemma}
Sean $\Psi, \Phi$ qubits. Luego, $|\braket{\Psi | \Phi}|^2 \leq 1$.
\end{lemma}

\begin{proof}
\begin{align*}
  |\braket{\Psi | \Phi}|^2
  &\le \braket{\Psi | \Psi} \braket{\Phi | \Phi}   \tag*{Desigualdad de Cauchy-Schwarz}
  \\ &= 1
\end{align*}

\end{proof}


\end{document}
