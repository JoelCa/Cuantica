\documentclass[a4paper,11pt]{article}
\usepackage[left=1.5cm,top=2.0cm,right=1.5cm,bottom=2.0cm]{geometry}
\usepackage[T1]{fontenc}
\usepackage[utf8]{inputenc}
\usepackage{lmodern,bm}
\usepackage[spanish]{babel}
\usepackage{hyperref}
\usepackage{caption}
\usepackage{subcaption}
\usepackage{amssymb}
\usepackage{enumitem}
\usepackage{amsthm}
\usepackage{braket}
\usepackage{stackrel}
\usepackage{qcircuit}
\usepackage{mathtools}

\title{}
\author{}

\begin{document}

\maketitle 

\thispagestyle{empty}
\begin{center}
\section*{Introducción a  la computación cuántica y fundamentos de lenguajes de programación} 
\subsection*{Ejercicios resueltos de la práctica 1}

\end{center}

\newpage{\pagestyle{empty}\cleardoublepage}

\newpage


\section{Ejercicios}

\subsection*{Ejercicio 10}
Consideramos el operador de medición $\{ M_{0}, M_{1}\}$, donde $M_{0}=\ket{+}\bra{+}$, $M_{1}=\ket{-}\bra{-}.$

\begin{enumerate}[label=\alph*)]
\item $\Psi = \frac{1}{3}\ket{0} + \frac{\sqrt{8}}{3}\ket{1}$. 

  \begin{itemize}
  \item $p(0) = \braket{\Psi | M_0^{\dag} M_0 | \Psi} \stackrel{(1)}{=} 
    \braket{\Psi | M_0 | \Psi} = \frac{1}{9}\cdot\frac{1}{2} + \frac{\sqrt{8}}{9}\cdot\frac{1}{2} + \frac{\sqrt{8}}{9}\cdot\frac{1}{2}
    + \frac{8}{9}\cdot\frac{1}{2} = \frac{\sqrt{8}}{9} + \frac{1}{2}$.
  \item $p(1) = \braket{\Psi | M_1^{\dag} M_1 | \Psi} \stackrel{(2)}{=} 
    \braket{\Psi | M_1 | \Psi} = \frac{1}{9}\cdot\frac{1}{2} - \frac{\sqrt{8}}{9}\cdot\frac{1}{2} - \frac{\sqrt{8}}{9}\cdot\frac{1}{2}
    + \frac{8}{9}\cdot\frac{1}{2} = - \frac{\sqrt{8}}{9} + \frac{1}{2}$.
  \end{itemize}

Si se aplica el proyector $M_0$, el sistema queda en el siguiente estado:

$\frac{M_0 \ket{\Psi}}{\sqrt{p(0)}} = \frac{1}{\sqrt{p(0)}} \Big( \braket{+|+}\Big( \frac{1}{3}\ket{0} + \frac{\sqrt{8}}{3}\ket{1} \Big) \Big) =
\frac{1}{\sqrt{p(0)}} \Big( \Big( \frac{1}{3\sqrt{2}}\ket{+} + \frac{\sqrt{8}}{3\sqrt{2}}\ket{+} \Big) \Big) =
\frac{1}{\sqrt{p(0)}} \Big( \frac{1}{6} + \frac{\sqrt{8}}{6} \Big) \Big( \ket{0} + \ket{1} \Big).$

Mientras que, si se aplica el proyector $M_1$, el sistema queda así:

$\frac{M_1 \ket{\Psi}}{\sqrt{p(1)}} = \frac{1}{\sqrt{p(1)}} \Big( \braket{-|-}\Big( \frac{1}{3}\ket{0} + \frac{\sqrt{8}}{3}\ket{1} \Big) \Big) =
\frac{1}{\sqrt{p(1)}} \Big( \Big( \frac{1}{3\sqrt{2}}\ket{-} - \frac{\sqrt{8}}{3\sqrt{2}}\ket{-} \Big) \Big) =
\frac{1}{\sqrt{p(1)}} \Big( \frac{1}{6} - \frac{\sqrt{8}}{6} \Big) \Big( \ket{0} - \ket{1} \Big).$

\item $\Psi = \frac{1}{\sqrt{2}} \Big( \ket{0} + \ket{1} \Big)$. 

  \begin{itemize}
  \item $p(0) = \braket{\Psi | M_0^{\dag} M_0 | \Psi} \stackrel{(1)}{=} 
    \braket{\Psi | M_0 | \Psi} = \frac{1}{2}\cdot\frac{1}{2} + \frac{1}{2}\cdot\frac{1}{2} + \frac{1}{2}\cdot\frac{1}{2} +
    \frac{1}{2}\cdot\frac{1}{2} = 1.$
  \item $p(1) = \braket{\Psi | M_1^{\dag} M_1 | \Psi} \stackrel{(2)}{=} 
    \braket{\Psi | M_1 | \Psi} = \frac{1}{2}\cdot\frac{1}{2} - \frac{1}{2}\cdot\frac{1}{2} - \frac{1}{2}\cdot\frac{1}{2} +
    \frac{1}{2}\cdot\frac{1}{2} = 0.$
  \end{itemize}
  Solo se aplica el proyector $M_0$, pues $p(0) = 1$. El sistema queda en el siguiente estado:

  $\frac{M_0 \ket{\Psi}}{\sqrt{p(0)}} = M_0 \ket{\Psi} = \braket{+|+} \frac{1}{\sqrt{2}}\Big(\ket{0} + \ket{1}\Big) = 
  \frac{1}{2} \ket{+} + \frac{1}{2} \ket{+} = \ket{+}.$

\item $\Psi = \ket{-}$

  \begin{itemize}
  \item $p(0) = \braket{- | M_0^{\dag} M_0 | -} \stackrel{(1)}{=} 
    \Big(\frac{1}{\sqrt{2}}\bra{0} - \frac{1}{\sqrt{2}}\bra{1} \Big) M_0  \Big(\frac{1}{\sqrt{2}}\ket{0} - \frac{1}{\sqrt{2}}\ket{1} \Big) =
    \frac{1}{2}\cdot\frac{1}{2} + \frac{1}{2}\cdot\frac{1}{2} + \frac{1}{2}\cdot\frac{1}{2} +
    \frac{1}{2}\cdot\frac{1}{2} = 1.$
  \item $p(1) = 0$
  \end{itemize}

  Solo se aplica el proyector $M_0$, pues $p(0) = 1$. El sistema queda en el siguiente estado.
  
  $\frac{M_0 \ket{\Psi}}{\sqrt{p(0)}} = M_0 \ket{-} = \braket{+|+}\ket{-} = \ket{+} \Big(\frac{1}{\sqrt{2}}\bra{0} + 
  \frac{1}{\sqrt{2}}\bra{1}\Big) \Big(\frac{1}{\sqrt{2}}\ket{0} - \frac{1}{\sqrt{2}}\ket{1}\Big) = 
  \ket{+} \Big( \frac{1}{2} - \frac{1}{2} \Big) = 0 \ket{+}.$
  
\end{enumerate}

Observaciones:

\begin{enumerate}[label=(\arabic*)]
\item $M_0$ hermitiana, y $M_0^2 = M_0$
\item $M_1$ hermitiana, y $M_1^2 = M_1$
\end{enumerate}

\subsection*{Ejercicio 13}

Sea la base $B = \{ \ket{+}, \ket{-}\}$. Un qubit $\Psi = \alpha\ket{+} + \beta \ket{-}$ está en superposición en la base $B$, si
$\alpha \neq 0$, y $\beta \neq 0$. 

\begin{enumerate}[label=\alph*)]
\item $\ket{0} = \frac{1}{\sqrt{2}}\Big( \ket{+} - \ket{-}\Big)$, está en superposición en la base $B$.
\item $\ket{+}$, no está en superposición en la base $B$.
\item $\frac{1}{\sqrt{2}}\Big( \ket{+} + \ket{-} \Big)$, está en superposición en $B$.
\item $\frac{1}{\sqrt{2}}\Big( \ket{+} - \ket{-} \Big)$, está en superposición en $B$.
\item $\frac{\sqrt{3}}{2}\ket{+} + \frac{1}{\sqrt{2}}\ket{-}$, está en superposición en $B$.
\item $\frac{1}{\sqrt{2}}\ket{0} - \frac{1}{\sqrt{2}}\ket{-} = \ket{-}$, no está en superposición en $B$.
\end{enumerate}

\subsection*{Ejercicio 17}

Circuito:

\vspace{0.5cm}

\Qcircuit @C=1em @R=1em {
\lstick{\ket{0}} & \gate{H} & \ctrl{1} & \qw & \qw \\
\lstick{\ket{0}} & \qw  & \targ & \ctrl{1} & \qw & \rstick{\frac{1}{\sqrt{2}}\ket{000} + \frac{1}{\sqrt{2}}\ket{111}}\\
\lstick{\ket{0}} & \qw  & \qw & \targ &  \qw
}

\vspace{0.5cm}

$\ket{000} \xrightarrow{H \otimes I \otimes I} \frac{1}{\sqrt{2}}\ket{000} + \frac{1}{\sqrt{2}}\ket{100}
\xrightarrow{CNOT \otimes I} \frac{1}{\sqrt{2}}\ket{000} + \frac{1}{\sqrt{2}}\ket{110}
\xrightarrow{I \otimes CNOT} \frac{1}{\sqrt{2}}\ket{000} + \frac{1}{\sqrt{2}}\ket{111}$

\end{document}