\documentclass[a4paper,11pt]{article}
\usepackage[left=1.5cm,top=2.0cm,right=1.5cm,bottom=2.0cm]{geometry}
\usepackage[T1]{fontenc}
\usepackage[utf8]{inputenc}
\usepackage{lmodern,bm}
\usepackage[spanish]{babel}
\usepackage{hyperref}
\usepackage{caption}
\usepackage{subcaption}
\usepackage{amssymb}
\usepackage{enumitem}
\usepackage{amsthm}
\usepackage{braket}
\usepackage{stackrel}
\usepackage{qcircuit}
\usepackage{mathtools}

\title{}
\author{}

\begin{document}

\maketitle 

\thispagestyle{empty}
\begin{center}
\section*{Introducción a  la computación cuántica y fundamentos de lenguajes de programación} 
\subsection*{Ejercicios resueltos de la práctica 1}

\end{center}

\newpage{\pagestyle{empty}\cleardoublepage}

\newpage

\DeclarePairedDelimiter\floor{\lfloor}{\rfloor}

\section*
{Ejercicios}

\subsection*{Ejercicio 10}
Consideramos el operador de medición $\{ M_{0}, M_{1}\}$, donde $M_{0}=\ket{+}\bra{+}$, $M_{1}=\ket{-}\bra{-}.$

\begin{enumerate}[label=\alph*)]
\item $\Psi = \frac{1}{3}\ket{0} + \frac{\sqrt{8}}{3}\ket{1}$. 

  \begin{itemize}
  \item $p(0) = \braket{\Psi | M_0^{\dag} M_0 | \Psi} \stackrel{(1)}{=} \frac{1}{9}\cdot\frac{1}{2} + 
    \frac{\sqrt{8}}{9}\cdot\frac{1}{2} + \frac{\sqrt{8}}{9}\cdot\frac{1}{2}
    + \frac{8}{9}\cdot\frac{1}{2} = \frac{\sqrt{8}}{9} + \frac{1}{2}$.
  \item $p(1) = \braket{\Psi | M_1^{\dag} M_1 | \Psi} \stackrel{(1)}{=} 
    \frac{1}{9}\cdot\frac{1}{2} - \frac{\sqrt{8}}{9}\cdot\frac{1}{2} - \frac{\sqrt{8}}{9}\cdot\frac{1}{2}
    + \frac{8}{9}\cdot\frac{1}{2} = - \frac{\sqrt{8}}{9} + \frac{1}{2}$.
  \end{itemize}

Si se aplica el proyector $M_0$, el sistema queda en el siguiente estado:

$\frac{M_0 \ket{\Psi}}{\sqrt{p(0)}} = \frac{1}{\sqrt{p(0)}} \Big( \ket{+}\bra{+}\Big( \frac{1}{3}\ket{0} + 
\frac{\sqrt{8}}{3}\ket{1} \Big) \Big) =
\frac{1}{\sqrt{p(0)}} \Big( \frac{1}{3\sqrt{2}}\ket{+} + \frac{\sqrt{8}}{3\sqrt{2}}\ket{+} \Big) =
\frac{1}{\sqrt{p(0)}} \Big( \frac{1}{6} + \frac{\sqrt{8}}{6} \Big) \Big( \ket{0} + \ket{1} \Big).$

Mientras que, si se aplica el proyector $M_1$, el sistema queda así:

$\frac{M_1 \ket{\Psi}}{\sqrt{p(1)}} = \frac{1}{\sqrt{p(1)}} \Big( \ket{-}\bra{-}\Big( \frac{1}{3}\ket{0} + 
\frac{\sqrt{8}}{3}\ket{1} \Big) \Big) =
\frac{1}{\sqrt{p(1)}} \Big( \frac{1}{3\sqrt{2}}\ket{-} - \frac{\sqrt{8}}{3\sqrt{2}}\ket{-} \Big) =
\frac{1}{\sqrt{p(1)}} \Big( \frac{1}{6} - \frac{\sqrt{8}}{6} \Big) \Big( \ket{0} - \ket{1} \Big).$

\item $\Psi = \frac{1}{\sqrt{2}} \Big( \ket{0} + \ket{1} \Big)$. 

  \begin{itemize}
  \item $p(0) = \braket{\Psi | M_0^{\dag} M_0 | \Psi} \stackrel{(1)}{=}
    \frac{1}{2}\cdot\frac{1}{2} + \frac{1}{2}\cdot\frac{1}{2} + \frac{1}{2}\cdot\frac{1}{2} +
    \frac{1}{2}\cdot\frac{1}{2} = 1.$
  \item $p(1) = \braket{\Psi | M_1^{\dag} M_1 | \Psi} \stackrel{(1)}{=}
    \frac{1}{2}\cdot\frac{1}{2} - \frac{1}{2}\cdot\frac{1}{2} - \frac{1}{2}\cdot\frac{1}{2} +
    \frac{1}{2}\cdot\frac{1}{2} = 0.$
  \end{itemize}
  Solo se aplica el proyector $M_0$, pues $p(0) = 1$. El sistema queda en el siguiente estado:

  $\frac{M_0 \ket{\Psi}}{\sqrt{p(0)}} = M_0 \ket{\Psi} = \ket{+}\bra{+} \frac{1}{\sqrt{2}}\Big(\ket{0} + \ket{1}\Big) = 
  \ket{+}\braket{+ | +} = \ket{+}.$

\item $\Psi = \ket{-}$

  \begin{itemize}
  \item $p(0) = \braket{- | M_0^{\dag} M_0 | -} = \braket{- | M_0 | -} =
    \braket{- | +}\braket{+ | -} = 0$. 
  \item $p(1) \braket{- | M_1^{\dag} M_1 | -} = \braket{- | M_1 | -} =
    \braket{- | -}\braket{- | -} = 1$
  \end{itemize}

  Solo se aplica el proyector $M_1$, pues $p(1) = 1$. El sistema queda en el siguiente estado:
  
  $\frac{M_1 \ket{\Psi}}{\sqrt{p(1)}} = M_1 \ket{-} = \ket{-} \braket{-|-} = \ket{-}$.
  
\end{enumerate}

Observación:

\begin{enumerate}[label=(\arabic*)]
\item $p(i) = \braket{\Psi | M_i^{\dag}M_i | \Psi} = \big(\alpha^*\bra{0} + \beta^*\bra{1}\big) M_i \big(\alpha\ket{0} + \beta\ket{1}\big) =
  |\alpha|^2\braket{0|M_i|0} + \alpha^*\beta\braket{0|M_i|1} + \alpha\beta^*\braket{1|M_i|0} + |\beta|^2\braket{1|M_i|1}$,
  siendo $M_i$ hermitiana, y $M_i^2 = M_i$
\end{enumerate}

\subsection*{Ejercicio 13}

Sea una base $B$. Un qubit $\Psi$ está en superposición en la base $B$, si no es un elemento de $B$.

Consideramos la base $B = \{ \ket{+}, \ket{-}\}$.

\begin{enumerate}[label=\alph*)]
\item $\ket{0} = \frac{1}{\sqrt{2}}\Big( \ket{+} - \ket{-}\Big)$, está en superposición en la base $B$.
\item $\ket{+}$, no está en superposición en la base $B$.
\item $\frac{1}{\sqrt{2}}\Big( \ket{+} + \ket{-} \Big)$, está en superposición en $B$.
\item $\frac{1}{\sqrt{2}}\Big( \ket{+} - \ket{-} \Big)$, está en superposición en $B$.
\item $\frac{\sqrt{3}}{2}\ket{+} + \frac{1}{\sqrt{2}}\ket{-}$, está en superposición en $B$.
\item $\frac{1}{\sqrt{2}}\ket{0} - \frac{1}{\sqrt{2}}\ket{-} = \ket{-}$, no está en superposición en $B$.
\end{enumerate}

\subsection*{Ejercicio 17}

Veamos que el siguiente circuito produce $\frac{1}{\sqrt{2}}\ket{000} + \frac{1}{\sqrt{2}}\ket{111}$,
para la entrada $\ket{000}$.

\vspace{0.5cm}

\Qcircuit @C=1em @R=1em {
\lstick{\ket{0}} & \gate{H} & \ctrl{1} & \qw & \qw \\
\lstick{\ket{0}} & \qw  & \targ & \ctrl{1} & \qw & \rstick{\frac{1}{\sqrt{2}}\ket{000} + \frac{1}{\sqrt{2}}\ket{111}}\\
\lstick{\ket{0}} & \qw  & \qw & \targ &  \qw
}

\vspace{0.5cm}

$\ket{000} \xrightarrow{H \otimes I \otimes I} \frac{1}{\sqrt{2}}\ket{000} + \frac{1}{\sqrt{2}}\ket{100}
\xrightarrow{CNOT \otimes I} \frac{1}{\sqrt{2}}\ket{000} + \frac{1}{\sqrt{2}}\ket{110}
\xrightarrow{I \otimes CNOT} \frac{1}{\sqrt{2}}\ket{000} + \frac{1}{\sqrt{2}}\ket{111}$.

\subsection*{Ejercicio 23}

Calcularemos la traza del algoritmo Deutsch para la función identidad.
Tenemos que $U_{id} \ket{x, y} = \ket{x, y \oplus x}.$

Luego, $\ket{ 01} \xrightarrow{H \otimes H} \frac{1}{\sqrt{2}} \Big( \ket{0} + \ket{1} \Big)
\frac{1}{\sqrt{2}} \Big( \ket{0} - \ket{1} \Big) = \frac{1}{\sqrt{2}} \Big( \ket{00} + \ket{10} - \ket{01} - \ket{11} \Big)
\xrightarrow{U_{id}} \frac{1}{\sqrt{2}} \Big( \ket{00} + \ket{11} - \ket{01} - \ket{10} \Big) =
\ket{--} \xrightarrow{H \otimes I} \ket{1-}$. 

Vamos a medir el primer qubit. Consideramos el operador de medición $\{ M_{0}, M_{1}\}$, donde $M_{0}=\ket{0}\bra{0}$, $M_{1}=\ket{1}\bra{1}$.

\begin{itemize}
\item $p(0) = \braket{1-| M_0 \otimes I|1-} = \frac{1}{2}(\bra{10} - \bra{11})(\ket{00}\bra{00} + \ket{01}\bra{01})(\ket{10} - \ket{11})=0$
\item $p(1) = \braket{1-| M_1 \otimes I|1-} = \frac{1}{2}(\bra{10} - \bra{11})(\ket{10}\bra{10} + \ket{11}\bra{11})(\ket{10} - \ket{11})=
\frac{1}{2}(\bra{10} - \bra{11})(\ket{10} - \ket{11}) = \frac{1}{2}(1+1)=1$
\end{itemize}

Se produce la siguiente evolución.

$\frac{(\ket{10}\bra{10} + \ket{11}\bra{11})\ket{1-}}{\sqrt{p(1)}} =  (\ket{10}\bra{10} + \ket{11}\bra{11}) \frac{1}{\sqrt{2}} (\ket{10} -\ket{11}) = 
\frac{1}{\sqrt{2}} (\ket{10} - \ket{11}) = \ket{1-}$.

La medición del primer qubit nos dará 1, con probabilidad 1.


\subsection*{Ejercicio 25}

\begin{enumerate}[label=\alph*)]
\item El número óptimo de iteraciones del algoritmo de Grover, para buscar un elemento en una lista de un millón de elementos, es
$\floor*{\frac{\pi}{4sen^{-1}(\sqrt{\frac{1}{2^{20}}})}} = 804$
\item La probabilidad de error con 804 iteraciones es $(2^{20}-1)m_{804}^{2} \simeq 2.43 \cdot 10^{-7} < \frac{1}{2^{20}}$
\item Con un algoritmo de búsqueda clásico sería necesario, en promedio, $\floor*{\frac{1}{N} \sum\limits_{i=1}^{N} i} = \floor*{\frac{N+1}{2}}$
pasos. En este caso, tendríamos que recorrer $500.000$ elementos, en promedio.
\end{enumerate}


\subsection*{Ejercicio 26}

Calculemos la probabilidad de que Eve sea detectado en el algoritmo BB84, si solo se transmite una clave de un bit.

Alice transmite $\ket{\Psi}$ en la base $B$. Luego, hay un $50\%$ de probabilidad de que Eve mida $\ket{\Psi}$ en
la base equivocada, dando lugar al qubit $\ket{\Phi}$. Luego, hay un $50\%$ de probabilidad de que Bob obtenga un qubit
distinto a $\ket{\Psi}$, midiendo el qubit $\ket{\Phi}$ en la base $B$. 
De esta forma, se detecta la presencia Eve, al existir un discrepancia en el hash que intercambian Alice y Bob.

Por lo tanto, hay un chance de $\frac{1}{4}$ de que Eve sea detectado, si solo se transmite una clave de longitud 1.

Eve solo puede pasar desapercibido si no es detectado en ninguna de las transmisiones donde Alice y Bob utilicen el mismo
esquema. 
Entonces, si Alice y Bob se ponen de acuerdo en una clave de longitud $N$, la probabilidad de que Eve no halla sido detectado es 
de $\big( \frac{3}{4} \big)^N$.

Por lo tanto, la probabilidad de que Eve sea detectado es de $1 - \big( \frac{3}{4} \big)^N$, donde $N$ es la cantidad de bits
que conforman la clave.  

Luego, para tener al menos $90 \%$ de chances de detectar a Eve, es necesario que Alice y Bob comparen al menos 9 bits,
pues $1 - \big( \frac{3}{4} \big)^9 \simeq 0.92$.


\end{document}








