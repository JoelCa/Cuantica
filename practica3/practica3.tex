\documentclass[a4paper,11pt]{article}
\usepackage[left=1.5cm,top=2.0cm,right=1.5cm,bottom=2.0cm]{geometry}
\usepackage[T1]{fontenc}
\usepackage[utf8]{inputenc}
\usepackage{lmodern,bm}
\usepackage[spanish]{babel}
\usepackage{hyperref}
\usepackage{caption}
\usepackage{subcaption}
\usepackage{amssymb}
\usepackage{enumitem}
\usepackage{braket}
\usepackage{stackrel}
\usepackage{cmll}
\usepackage{bussproofs}

\title{}
\author{}

\begin{document}

\maketitle 

\thispagestyle{empty}
\begin{center}
\section*{Introducción a  la computación cuántica y fundamentos de lenguajes de programación} 
\subsection*{Ejercicios resueltos de la práctica 3}
\subsubsection*{Joel Catacora}

\end{center}

\newpage{\pagestyle{empty}\cleardoublepage}

\newpage

\section*
{Ejercicios}

\subsection*{Ejercicio 42}

\begin{itemize}
\item Regla asociada a $ax_v$. Tenemos que probar que, $A \vdash A$.

  \begin{prooftree}
    \AxiomC{}
    \RightLabel{\scriptsize{$ax$}}
    \UnaryInfC{$A \vdash A$}
  \end{prooftree}

\item Regla asociadad a $ax_c$. Tenemos que probar que, $\vdash !1$.
  
  \begin{prooftree}
    \AxiomC{}
    \RightLabel{\scriptsize{$1_r$}}
    \UnaryInfC{$\vdash 1$}
    \RightLabel{\scriptsize{$\oc_r$}}
    \UnaryInfC{$\vdash \oc 1$}
  \end{prooftree}

\item Regla asociada a $\multimap_i$. Tenemos que probar que si $\Gamma, A \vdash B$; entonces $\Gamma \vdash \neg A \parr B$.

  \begin{prooftree}
    \AxiomC{$\Gamma, A \vdash B$}
    \RightLabel{\scriptsize{$\neg_r$}}
    \UnaryInfC{$\Gamma \vdash \neg A, B$}
    \RightLabel{\scriptsize{$\parr_r$}}
    \UnaryInfC{$ \Gamma \vdash \neg A \parr B$}
  \end{prooftree}

\item Regla asociada a $\multimap_e$. Probemos que si $\Gamma \vdash \neg A \parr B$; y $\Delta \vdash A$; entonces
  $\Gamma, \Delta \vdash B$.
 
  \begin{prooftree}
    \AxiomC{$\Gamma \vdash \neg A \parr B$}

    \AxiomC{$\Delta \vdash A$}
    \RightLabel{\scriptsize{$\neg_l$}}
    \UnaryInfC{$\Delta, \neg A \vdash$}

    \AxiomC{}
    \RightLabel{\scriptsize{$ax$}}
    \UnaryInfC{$B \vdash B$}
    \RightLabel{\scriptsize{$\parr_l$}}
    \BinaryInfC{$\Delta, \neg A \parr B \vdash B$}

    \RightLabel{\scriptsize{$cut$}}
    \BinaryInfC{$ \Gamma, \Delta \vdash B$}
  \end{prooftree}

\item Regla asociada a $\odot$. Probemos que si $\Gamma \vdash \oc 1$; y $\Delta \vdash \oc 1$; entonces 
  $\Gamma, \Delta \vdash \oc 1$.

  \begin{prooftree}
    \AxiomC{$\Gamma \vdash \oc 1$}
    \RightLabel{\scriptsize{$1_l$}}
    \UnaryInfC{$\Gamma, 1 \vdash \oc 1 $}
    \RightLabel{\scriptsize{$D_\oc$}}
    \UnaryInfC{$\Gamma, \oc 1 \vdash \oc 1 $}

    \AxiomC{$\Delta \vdash \oc 1$}
    \RightLabel{\scriptsize{$cut$}}
    \BinaryInfC{$ \Gamma, \Delta \vdash \oc 1$}
  \end{prooftree}

\item Regla asociada a $ifz$. Veamos que $\Gamma \vdash \oc 1$; y $\Delta \vdash A$; entonces 
  $\Gamma, \Delta \vdash A$

  \begin{prooftree}
    \AxiomC{$\Gamma \vdash \oc 1$}
    \AxiomC{$\Delta \vdash A$}
    \RightLabel{\scriptsize{$W_{\oc}$}}
    \UnaryInfC{$\Delta, \oc 1 \vdash A $}
    \RightLabel{\scriptsize{$cut$}}
    \BinaryInfC{$ \Gamma, \Delta \vdash A$}
  \end{prooftree}

\item Regla asociada a $let$. Probemos que $\Gamma, A \vdash B$; y $\Delta \vdash A$; entonces
  $\Gamma, \Delta \vdash B$.

  \begin{prooftree}
    \AxiomC{$\Gamma, A \vdash B$}
    \AxiomC{$\Delta \vdash A$}
    \RightLabel{\scriptsize{$cut$}}
    \BinaryInfC{$\Gamma, \Delta \vdash B$}
  \end{prooftree}

\item Regla asociada a $W$. Luego, veamos que si $\Gamma \vdash A$; entonces $\Gamma, \oc B \vdash A$.

  \begin{prooftree}
    \AxiomC{$\Gamma \vdash A$}
    \RightLabel{\scriptsize{$W_{\oc}$}}
    \UnaryInfC{$\Gamma, \oc B \vdash A$}
  \end{prooftree}

\item Regla asociada a $C$. Luego, tenemos que si $\Gamma, \oc B, \oc B \vdash A$, entonces
  $\Gamma, \oc B \vdash A$.

  \begin{prooftree}
    \AxiomC{$\Gamma, \oc B, \oc B \vdash A$}
    \RightLabel{\scriptsize{$C_{\oc}$}}
    \UnaryInfC{$\Gamma, \oc B \vdash A$}
  \end{prooftree}

\end{itemize}


\subsection*{Ejercicio 43}

Daremos la derivación de tipos para el término $\lambda x. x + x$.


  \begin{prooftree}
    \AxiomC{}
    \RightLabel{\scriptsize{$ax_v$}}
    \UnaryInfC{$x : \oc \textsf{nat} \vdash x : \oc \textsf{nat}$}
    \AxiomC{}
    \RightLabel{\scriptsize{$ax_v$}}
    \UnaryInfC{$y : \oc \textsf{nat} \vdash y : \oc \textsf{nat}$}
    \RightLabel{\scriptsize{$\odot$}}
    \BinaryInfC{$x : \oc \textsf{nat}, y : \oc \textsf{nat} \vdash x + y : \oc \textsf{nat}$}
    \RightLabel{\scriptsize{$C$}}
    \UnaryInfC{$x : \oc \textsf{nat} \vdash x + x : \oc \textsf{nat}$}
    \RightLabel{\scriptsize{$\multimap_i$}}
    \UnaryInfC{$\vdash \lambda x. x + x : \oc \textsf{nat} \multimap \oc \textsf{nat}$}
  \end{prooftree}

Luego, $\lambda x. x + x : \oc \textsf{nat} \multimap \oc \textsf{nat}$.
\end{document}








